% Options for packages loaded elsewhere
\PassOptionsToPackage{unicode}{hyperref}
\PassOptionsToPackage{hyphens}{url}
%
\documentclass[
]{article}
\usepackage{amsmath,amssymb}
\usepackage{lmodern}
\usepackage{iftex}
\ifPDFTeX
  \usepackage[T1]{fontenc}
  \usepackage[utf8]{inputenc}
  \usepackage{textcomp} % provide euro and other symbols
\else % if luatex or xetex
  \usepackage{unicode-math}
  \defaultfontfeatures{Scale=MatchLowercase}
  \defaultfontfeatures[\rmfamily]{Ligatures=TeX,Scale=1}
\fi
% Use upquote if available, for straight quotes in verbatim environments
\IfFileExists{upquote.sty}{\usepackage{upquote}}{}
\IfFileExists{microtype.sty}{% use microtype if available
  \usepackage[]{microtype}
  \UseMicrotypeSet[protrusion]{basicmath} % disable protrusion for tt fonts
}{}
\makeatletter
\@ifundefined{KOMAClassName}{% if non-KOMA class
  \IfFileExists{parskip.sty}{%
    \usepackage{parskip}
  }{% else
    \setlength{\parindent}{0pt}
    \setlength{\parskip}{6pt plus 2pt minus 1pt}}
}{% if KOMA class
  \KOMAoptions{parskip=half}}
\makeatother
\usepackage{xcolor}
\usepackage[margin=2.54cm]{geometry}
\usepackage{color}
\usepackage{fancyvrb}
\newcommand{\VerbBar}{|}
\newcommand{\VERB}{\Verb[commandchars=\\\{\}]}
\DefineVerbatimEnvironment{Highlighting}{Verbatim}{commandchars=\\\{\}}
% Add ',fontsize=\small' for more characters per line
\usepackage{framed}
\definecolor{shadecolor}{RGB}{248,248,248}
\newenvironment{Shaded}{\begin{snugshade}}{\end{snugshade}}
\newcommand{\AlertTok}[1]{\textcolor[rgb]{0.94,0.16,0.16}{#1}}
\newcommand{\AnnotationTok}[1]{\textcolor[rgb]{0.56,0.35,0.01}{\textbf{\textit{#1}}}}
\newcommand{\AttributeTok}[1]{\textcolor[rgb]{0.77,0.63,0.00}{#1}}
\newcommand{\BaseNTok}[1]{\textcolor[rgb]{0.00,0.00,0.81}{#1}}
\newcommand{\BuiltInTok}[1]{#1}
\newcommand{\CharTok}[1]{\textcolor[rgb]{0.31,0.60,0.02}{#1}}
\newcommand{\CommentTok}[1]{\textcolor[rgb]{0.56,0.35,0.01}{\textit{#1}}}
\newcommand{\CommentVarTok}[1]{\textcolor[rgb]{0.56,0.35,0.01}{\textbf{\textit{#1}}}}
\newcommand{\ConstantTok}[1]{\textcolor[rgb]{0.00,0.00,0.00}{#1}}
\newcommand{\ControlFlowTok}[1]{\textcolor[rgb]{0.13,0.29,0.53}{\textbf{#1}}}
\newcommand{\DataTypeTok}[1]{\textcolor[rgb]{0.13,0.29,0.53}{#1}}
\newcommand{\DecValTok}[1]{\textcolor[rgb]{0.00,0.00,0.81}{#1}}
\newcommand{\DocumentationTok}[1]{\textcolor[rgb]{0.56,0.35,0.01}{\textbf{\textit{#1}}}}
\newcommand{\ErrorTok}[1]{\textcolor[rgb]{0.64,0.00,0.00}{\textbf{#1}}}
\newcommand{\ExtensionTok}[1]{#1}
\newcommand{\FloatTok}[1]{\textcolor[rgb]{0.00,0.00,0.81}{#1}}
\newcommand{\FunctionTok}[1]{\textcolor[rgb]{0.00,0.00,0.00}{#1}}
\newcommand{\ImportTok}[1]{#1}
\newcommand{\InformationTok}[1]{\textcolor[rgb]{0.56,0.35,0.01}{\textbf{\textit{#1}}}}
\newcommand{\KeywordTok}[1]{\textcolor[rgb]{0.13,0.29,0.53}{\textbf{#1}}}
\newcommand{\NormalTok}[1]{#1}
\newcommand{\OperatorTok}[1]{\textcolor[rgb]{0.81,0.36,0.00}{\textbf{#1}}}
\newcommand{\OtherTok}[1]{\textcolor[rgb]{0.56,0.35,0.01}{#1}}
\newcommand{\PreprocessorTok}[1]{\textcolor[rgb]{0.56,0.35,0.01}{\textit{#1}}}
\newcommand{\RegionMarkerTok}[1]{#1}
\newcommand{\SpecialCharTok}[1]{\textcolor[rgb]{0.00,0.00,0.00}{#1}}
\newcommand{\SpecialStringTok}[1]{\textcolor[rgb]{0.31,0.60,0.02}{#1}}
\newcommand{\StringTok}[1]{\textcolor[rgb]{0.31,0.60,0.02}{#1}}
\newcommand{\VariableTok}[1]{\textcolor[rgb]{0.00,0.00,0.00}{#1}}
\newcommand{\VerbatimStringTok}[1]{\textcolor[rgb]{0.31,0.60,0.02}{#1}}
\newcommand{\WarningTok}[1]{\textcolor[rgb]{0.56,0.35,0.01}{\textbf{\textit{#1}}}}
\usepackage{graphicx}
\makeatletter
\def\maxwidth{\ifdim\Gin@nat@width>\linewidth\linewidth\else\Gin@nat@width\fi}
\def\maxheight{\ifdim\Gin@nat@height>\textheight\textheight\else\Gin@nat@height\fi}
\makeatother
% Scale images if necessary, so that they will not overflow the page
% margins by default, and it is still possible to overwrite the defaults
% using explicit options in \includegraphics[width, height, ...]{}
\setkeys{Gin}{width=\maxwidth,height=\maxheight,keepaspectratio}
% Set default figure placement to htbp
\makeatletter
\def\fps@figure{htbp}
\makeatother
\setlength{\emergencystretch}{3em} % prevent overfull lines
\providecommand{\tightlist}{%
  \setlength{\itemsep}{0pt}\setlength{\parskip}{0pt}}
\setcounter{secnumdepth}{-\maxdimen} % remove section numbering
\ifLuaTeX
  \usepackage{selnolig}  % disable illegal ligatures
\fi
\IfFileExists{bookmark.sty}{\usepackage{bookmark}}{\usepackage{hyperref}}
\IfFileExists{xurl.sty}{\usepackage{xurl}}{} % add URL line breaks if available
\urlstyle{same} % disable monospaced font for URLs
\hypersetup{
  pdftitle={Assignment 4: Data Wrangling},
  pdfauthor={Kaichun Yang},
  hidelinks,
  pdfcreator={LaTeX via pandoc}}

\title{Assignment 4: Data Wrangling}
\author{Kaichun Yang}
\date{}

\begin{document}
\maketitle

\hypertarget{overview}{%
\subsection{OVERVIEW}\label{overview}}

This exercise accompanies the lessons in Environmental Data Analytics on
Data Wrangling

\hypertarget{directions}{%
\subsection{Directions}\label{directions}}

\begin{enumerate}
\def\labelenumi{\arabic{enumi}.}
\tightlist
\item
  Rename this file
  \texttt{\textless{}FirstLast\textgreater{}\_A03\_DataExploration.Rmd}
  (replacing \texttt{\textless{}FirstLast\textgreater{}} with your first
  and last name).
\item
  Change ``Student Name'' on line 3 (above) with your name.
\item
  Work through the steps, \textbf{creating code and output} that fulfill
  each instruction.
\item
  Be sure to \textbf{answer the questions} in this assignment document.
\item
  When you have completed the assignment, \textbf{Knit} the text and
  code into a single PDF file.
\end{enumerate}

The completed exercise is due on Friday, Oct7th @ 5:00pm.

\hypertarget{set-up-your-session}{%
\subsection{Set up your session}\label{set-up-your-session}}

\begin{enumerate}
\def\labelenumi{\arabic{enumi}.}
\tightlist
\item
  Check your working directory, load the \texttt{tidyverse} and
  \texttt{lubridate} packages, and upload all four raw data files
  associated with the EPA Air dataset, being sure to set string columns
  to be read in a factors. See the README file for the EPA air datasets
  for more information (especially if you have not worked with air
  quality data previously).
\end{enumerate}

\begin{Shaded}
\begin{Highlighting}[]
\CommentTok{\# 1}
\FunctionTok{library}\NormalTok{(tidyverse)}
\end{Highlighting}
\end{Shaded}

\begin{verbatim}
## -- Attaching packages --------------------------------------- tidyverse 1.3.2 --
## v ggplot2 3.3.6      v purrr   0.3.4 
## v tibble  3.1.8      v dplyr   1.0.10
## v tidyr   1.2.1      v stringr 1.4.1 
## v readr   2.1.2      v forcats 0.5.2 
## -- Conflicts ------------------------------------------ tidyverse_conflicts() --
## x dplyr::filter() masks stats::filter()
## x dplyr::lag()    masks stats::lag()
\end{verbatim}

\begin{Shaded}
\begin{Highlighting}[]
\FunctionTok{library}\NormalTok{(lubridate)}
\end{Highlighting}
\end{Shaded}

\begin{verbatim}
## 
## 载入程辑包:'lubridate'
## 
## The following objects are masked from 'package:base':
## 
##     date, intersect, setdiff, union
\end{verbatim}

\begin{Shaded}
\begin{Highlighting}[]
\NormalTok{EPA\_O3\_2018 }\OtherTok{=} \FunctionTok{read.csv}\NormalTok{(}\AttributeTok{file =} \StringTok{"E:/EDA{-}Fall2022/Data/Raw/EPAair\_O3\_NC2018\_raw.csv"}\NormalTok{)}
\NormalTok{EPA\_O3\_2019 }\OtherTok{=} \FunctionTok{read.csv}\NormalTok{(}\AttributeTok{file =} \StringTok{"E:/EDA{-}Fall2022/Data/Raw/EPAair\_O3\_NC2019\_raw.csv"}\NormalTok{)}
\NormalTok{EPA\_PM25\_2018 }\OtherTok{=} \FunctionTok{read.csv}\NormalTok{(}\AttributeTok{file =} \StringTok{"E:/EDA{-}Fall2022/Data/Raw/EPAair\_PM25\_NC2018\_raw.csv"}\NormalTok{)}
\NormalTok{EPA\_PM25\_2019 }\OtherTok{=} \FunctionTok{read.csv}\NormalTok{(}\AttributeTok{file =} \StringTok{"E:/EDA{-}Fall2022/Data/Raw/EPAair\_PM25\_NC2019\_raw.csv"}\NormalTok{)}
\end{Highlighting}
\end{Shaded}

\begin{enumerate}
\def\labelenumi{\arabic{enumi}.}
\setcounter{enumi}{1}
\tightlist
\item
  Explore the dimensions, column names, and structure of the datasets.
\end{enumerate}

\hypertarget{wrangle-individual-datasets-to-create-processed-files.}{%
\subsection{Wrangle individual datasets to create processed
files.}\label{wrangle-individual-datasets-to-create-processed-files.}}

\begin{enumerate}
\def\labelenumi{\arabic{enumi}.}
\setcounter{enumi}{2}
\tightlist
\item
  Change date to date
\item
  Select the following columns: Date, DAILY\_AQI\_VALUE, Site.Name,
  AQS\_PARAMETER\_DESC, COUNTY, SITE\_LATITUDE, SITE\_LONGITUDE
\item
  For the PM2.5 datasets, fill all cells in AQS\_PARAMETER\_DESC with
  ``PM2.5'' (all cells in this column should be identical).
\item
  Save all four processed datasets in the Processed folder. Use the same
  file names as the raw files but replace ``raw'' with ``processed''.
\end{enumerate}

\begin{Shaded}
\begin{Highlighting}[]
\CommentTok{\# 3}
\NormalTok{EPA\_O3\_2018}\SpecialCharTok{$}\NormalTok{Date }\OtherTok{\textless{}{-}} \FunctionTok{as.Date}\NormalTok{(EPA\_O3\_2018}\SpecialCharTok{$}\NormalTok{Date, }\StringTok{"\%m/\%d/\%Y"}\NormalTok{)}
\NormalTok{EPA\_O3\_2019}\SpecialCharTok{$}\NormalTok{Date }\OtherTok{\textless{}{-}} \FunctionTok{as.Date}\NormalTok{(EPA\_O3\_2019}\SpecialCharTok{$}\NormalTok{Date, }\StringTok{"\%m/\%d/\%Y"}\NormalTok{)}
\NormalTok{EPA\_PM25\_2018}\SpecialCharTok{$}\NormalTok{Date }\OtherTok{\textless{}{-}} \FunctionTok{as.Date}\NormalTok{(EPA\_PM25\_2018}\SpecialCharTok{$}\NormalTok{Date, }\StringTok{"\%m/\%d/\%Y"}\NormalTok{)}
\NormalTok{EPA\_PM25\_2019}\SpecialCharTok{$}\NormalTok{Date }\OtherTok{\textless{}{-}} \FunctionTok{as.Date}\NormalTok{(EPA\_PM25\_2019}\SpecialCharTok{$}\NormalTok{Date, }\StringTok{"\%m/\%d/\%Y"}\NormalTok{)}

\CommentTok{\# 4}
\NormalTok{EPA\_O3\_2018\_S }\OtherTok{\textless{}{-}} \FunctionTok{select}\NormalTok{(EPA\_O3\_2018, Date, DAILY\_AQI\_VALUE, Site.Name, AQS\_PARAMETER\_DESC,}
\NormalTok{    COUNTY, SITE\_LATITUDE, SITE\_LONGITUDE)}
\NormalTok{EPA\_O3\_2019\_S }\OtherTok{\textless{}{-}} \FunctionTok{select}\NormalTok{(EPA\_O3\_2019, Date, DAILY\_AQI\_VALUE, Site.Name, AQS\_PARAMETER\_DESC,}
\NormalTok{    COUNTY, SITE\_LATITUDE, SITE\_LONGITUDE)}
\NormalTok{EPA\_PM25\_2018\_s }\OtherTok{\textless{}{-}} \FunctionTok{select}\NormalTok{(EPA\_PM25\_2018, Date, DAILY\_AQI\_VALUE, Site.Name, AQS\_PARAMETER\_DESC,}
\NormalTok{    COUNTY, SITE\_LATITUDE, SITE\_LONGITUDE)}
\NormalTok{EPA\_PM25\_2019\_s }\OtherTok{\textless{}{-}} \FunctionTok{select}\NormalTok{(EPA\_PM25\_2019, Date, DAILY\_AQI\_VALUE, Site.Name, AQS\_PARAMETER\_DESC,}
\NormalTok{    COUNTY, SITE\_LATITUDE, SITE\_LONGITUDE)}

\CommentTok{\# 5}
\NormalTok{EPA\_PM25\_2018\_s}\SpecialCharTok{$}\NormalTok{AQS\_PARAMETER\_DESC }\OtherTok{\textless{}{-}} \StringTok{"PM2.5"}
\NormalTok{EPA\_PM25\_2019\_s}\SpecialCharTok{$}\NormalTok{AQS\_PARAMETER\_DESC }\OtherTok{\textless{}{-}} \StringTok{"PM2.5"}

\CommentTok{\# 6}
\FunctionTok{write.csv}\NormalTok{(EPA\_O3\_2018\_S, }\AttributeTok{row.names =} \ConstantTok{FALSE}\NormalTok{, }\AttributeTok{file =} \StringTok{"E:/EDA{-}Fall2022/Data/Raw/EPAair\_O3\_NC2018\_processed.csv"}\NormalTok{)}
\FunctionTok{write.csv}\NormalTok{(EPA\_O3\_2019\_S, }\AttributeTok{row.names =} \ConstantTok{FALSE}\NormalTok{, }\AttributeTok{file =} \StringTok{"E:/EDA{-}Fall2022/Data/Raw/EPAair\_O3\_NC2019\_processed.csv"}\NormalTok{)}
\FunctionTok{write.csv}\NormalTok{(EPA\_PM25\_2018\_s, }\AttributeTok{row.names =} \ConstantTok{FALSE}\NormalTok{, }\AttributeTok{file =} \StringTok{"E:/EDA{-}Fall2022/Data/Raw/EPAair\_PM25\_NC2018\_processed.csv"}\NormalTok{)}
\FunctionTok{write.csv}\NormalTok{(EPA\_PM25\_2019\_s, }\AttributeTok{row.names =} \ConstantTok{FALSE}\NormalTok{, }\AttributeTok{file =} \StringTok{"E:/EDA{-}Fall2022/Data/Raw/EPAair\_PM25\_NC2019\_processed.csv"}\NormalTok{)}
\end{Highlighting}
\end{Shaded}

\hypertarget{combine-datasets}{%
\subsection{Combine datasets}\label{combine-datasets}}

\begin{enumerate}
\def\labelenumi{\arabic{enumi}.}
\setcounter{enumi}{6}
\tightlist
\item
  Combine the four datasets with \texttt{rbind}. Make sure your column
  names are identical prior to running this code.
\item
  Wrangle your new dataset with a pipe function (\%\textgreater\%) so
  that it fills the following conditions:
\end{enumerate}

\begin{itemize}
\item
  Include all sites that the four data frames have in common: ``Linville
  Falls'', ``Durham Armory'', ``Leggett'', ``Hattie Avenue'', ``Clemmons
  Middle'', ``Mendenhall School'', ``Frying Pan Mountain'', ``West
  Johnston Co.'', ``Garinger High School'', ``Castle Hayne'', ``Pitt
  Agri. Center'', ``Bryson City'', ``Millbrook School'' (the function
  \texttt{intersect} can figure out common factor levels)
\item
  Some sites have multiple measurements per day. Use the
  split-apply-combine strategy to generate daily means: group by date,
  site, aqs parameter, and county. Take the mean of the AQI value,
  latitude, and longitude.
\item
  Add columns for ``Month'' and ``Year'' by parsing your ``Date'' column
  (hint: \texttt{lubridate} package)
\item
  Hint: the dimensions of this dataset should be 14,752 x 9.
\end{itemize}

\begin{enumerate}
\def\labelenumi{\arabic{enumi}.}
\setcounter{enumi}{8}
\tightlist
\item
  Spread your datasets such that AQI values for ozone and PM2.5 are in
  separate columns. Each location on a specific date should now occupy
  only one row.
\item
  Call up the dimensions of your new tidy dataset.
\item
  Save your processed dataset with the following file name:
  ``\texttt{EPAair\_O3\_PM25\_NC1718\_Processed.csv}''
\end{enumerate}

\begin{Shaded}
\begin{Highlighting}[]
\CommentTok{\# intersect figure out common factor level}

\FunctionTok{library}\NormalTok{(dplyr)}
\FunctionTok{library}\NormalTok{(lubridate)}

\CommentTok{\# 7}
\NormalTok{EPA\_data }\OtherTok{\textless{}{-}} \FunctionTok{rbind}\NormalTok{(EPA\_O3\_2018\_S, EPA\_O3\_2019\_S, EPA\_PM25\_2018\_s, EPA\_PM25\_2019\_s)}

\CommentTok{\# 8}
\NormalTok{EPA\_data\_2 }\OtherTok{\textless{}{-}}\NormalTok{ EPA\_data }\SpecialCharTok{\%\textgreater{}\%}
    \FunctionTok{filter}\NormalTok{(Site.Name }\SpecialCharTok{==} \StringTok{"Linville Falls"} \SpecialCharTok{|}\NormalTok{ Site.Name }\SpecialCharTok{==} \StringTok{"Durham Armory"} \SpecialCharTok{|}\NormalTok{ Site.Name }\SpecialCharTok{==}
        \StringTok{"Leggett"} \SpecialCharTok{|}\NormalTok{ Site.Name }\SpecialCharTok{==} \StringTok{"Hattie Avenue"} \SpecialCharTok{|}\NormalTok{ Site.Name }\SpecialCharTok{==} \StringTok{"Clemmons Middle"} \SpecialCharTok{|}
\NormalTok{        Site.Name }\SpecialCharTok{==} \StringTok{"Mendenhall School"} \SpecialCharTok{|}\NormalTok{ Site.Name }\SpecialCharTok{==} \StringTok{"Frying Pan Mountain"} \SpecialCharTok{|}\NormalTok{ Site.Name }\SpecialCharTok{==}
        \StringTok{"West Johnston Co."} \SpecialCharTok{|}\NormalTok{ Site.Name }\SpecialCharTok{==} \StringTok{"Garinger High School"} \SpecialCharTok{|}\NormalTok{ Site.Name }\SpecialCharTok{==}
        \StringTok{"Castle Hayne"} \SpecialCharTok{|}\NormalTok{ Site.Name }\SpecialCharTok{==} \StringTok{"Pitt Agri. Center"} \SpecialCharTok{|}\NormalTok{ Site.Name }\SpecialCharTok{==} \StringTok{"Bryson City"} \SpecialCharTok{|}
\NormalTok{        Site.Name }\SpecialCharTok{==} \StringTok{"Millbrook School"}\NormalTok{) }\SpecialCharTok{\%\textgreater{}\%}
    \FunctionTok{group\_by}\NormalTok{(Date, Site.Name, AQS\_PARAMETER\_DESC, COUNTY) }\SpecialCharTok{\%\textgreater{}\%}
    \FunctionTok{summarise}\NormalTok{(}\AttributeTok{meanaqi =} \FunctionTok{mean}\NormalTok{(DAILY\_AQI\_VALUE), }\AttributeTok{meanlat =} \FunctionTok{mean}\NormalTok{(SITE\_LATITUDE), }\AttributeTok{meanlog =} \FunctionTok{mean}\NormalTok{(SITE\_LONGITUDE),}
        \AttributeTok{.groups =} \StringTok{"keep"}\NormalTok{) }\SpecialCharTok{\%\textgreater{}\%}
    \FunctionTok{mutate}\NormalTok{(}\AttributeTok{Year =} \FunctionTok{year}\NormalTok{(Date), }\AttributeTok{Month =} \FunctionTok{month}\NormalTok{(Date))}
\FunctionTok{print}\NormalTok{(EPA\_data\_2)}
\end{Highlighting}
\end{Shaded}

\begin{verbatim}
## # A tibble: 14,752 x 9
## # Groups:   Date, Site.Name, AQS_PARAMETER_DESC, COUNTY [14,752]
##    Date       Site.Name       AQS_P~1 COUNTY meanaqi meanlat meanlog  Year Month
##    <date>     <chr>           <chr>   <chr>    <dbl>   <dbl>   <dbl> <dbl> <dbl>
##  1 2018-01-01 Bryson City     PM2.5   Swain       35    35.4   -83.4  2018     1
##  2 2018-01-01 Castle Hayne    PM2.5   New H~      13    34.4   -77.8  2018     1
##  3 2018-01-01 Clemmons Middle PM2.5   Forsy~      24    36.0   -80.3  2018     1
##  4 2018-01-01 Durham Armory   PM2.5   Durham      31    36.0   -78.9  2018     1
##  5 2018-01-01 Garinger High ~ Ozone   Meckl~      32    35.2   -80.8  2018     1
##  6 2018-01-01 Garinger High ~ PM2.5   Meckl~      20    35.2   -80.8  2018     1
##  7 2018-01-01 Hattie Avenue   PM2.5   Forsy~      22    36.1   -80.2  2018     1
##  8 2018-01-01 Leggett         PM2.5   Edgec~      14    36.0   -77.6  2018     1
##  9 2018-01-01 Millbrook Scho~ Ozone   Wake        34    35.9   -78.6  2018     1
## 10 2018-01-01 Millbrook Scho~ PM2.5   Wake        28    35.9   -78.6  2018     1
## # ... with 14,742 more rows, and abbreviated variable name
## #   1: AQS_PARAMETER_DESC
\end{verbatim}

\begin{Shaded}
\begin{Highlighting}[]
\CommentTok{\# 9}
\NormalTok{EPA\_data\_3 }\OtherTok{\textless{}{-}}\NormalTok{ EPA\_data\_2 }\SpecialCharTok{\%\textgreater{}\%}
    \FunctionTok{pivot\_wider}\NormalTok{(}\AttributeTok{names\_from =} \StringTok{"AQS\_PARAMETER\_DESC"}\NormalTok{, }\AttributeTok{values\_from =} \StringTok{"meanaqi"}\NormalTok{)}
\FunctionTok{print}\NormalTok{(EPA\_data\_3)}
\end{Highlighting}
\end{Shaded}

\begin{verbatim}
## # A tibble: 8,976 x 9
## # Groups:   Date, Site.Name, COUNTY [8,976]
##    Date       Site.Name           COUNTY meanlat meanlog  Year Month PM2.5 Ozone
##    <date>     <chr>               <chr>    <dbl>   <dbl> <dbl> <dbl> <dbl> <dbl>
##  1 2018-01-01 Bryson City         Swain     35.4   -83.4  2018     1    35    NA
##  2 2018-01-01 Castle Hayne        New H~    34.4   -77.8  2018     1    13    NA
##  3 2018-01-01 Clemmons Middle     Forsy~    36.0   -80.3  2018     1    24    NA
##  4 2018-01-01 Durham Armory       Durham    36.0   -78.9  2018     1    31    NA
##  5 2018-01-01 Garinger High Scho~ Meckl~    35.2   -80.8  2018     1    20    32
##  6 2018-01-01 Hattie Avenue       Forsy~    36.1   -80.2  2018     1    22    NA
##  7 2018-01-01 Leggett             Edgec~    36.0   -77.6  2018     1    14    NA
##  8 2018-01-01 Millbrook School    Wake      35.9   -78.6  2018     1    28    34
##  9 2018-01-01 Pitt Agri. Center   Pitt      35.6   -77.4  2018     1    15    NA
## 10 2018-01-01 West Johnston Co.   Johns~    35.6   -78.5  2018     1    24    NA
## # ... with 8,966 more rows
\end{verbatim}

\begin{Shaded}
\begin{Highlighting}[]
\CommentTok{\# 10}
\FunctionTok{dim}\NormalTok{(EPA\_data\_3)}
\end{Highlighting}
\end{Shaded}

\begin{verbatim}
## [1] 8976    9
\end{verbatim}

\begin{Shaded}
\begin{Highlighting}[]
\CommentTok{\# 11}
\FunctionTok{write.csv}\NormalTok{(EPA\_data\_3, }\AttributeTok{row.names =} \ConstantTok{FALSE}\NormalTok{, }\AttributeTok{file =} \StringTok{"E:/EDA{-}Fall2022/Data/Raw/EPAair\_O3\_PM25\_NC1718\_Processed.csv"}\NormalTok{)}
\end{Highlighting}
\end{Shaded}

\hypertarget{generate-summary-tables}{%
\subsection{Generate summary tables}\label{generate-summary-tables}}

\begin{enumerate}
\def\labelenumi{\arabic{enumi}.}
\setcounter{enumi}{11}
\item
  Use the split-apply-combine strategy to generate a summary data frame.
  Data should be grouped by site, month, and year. Generate the mean AQI
  values for ozone and PM2.5 for each group. Then, add a pipe to remove
  instances where a month and year are not available (use the function
  \texttt{drop\_na} in your pipe).
\item
  Call up the dimensions of the summary dataset.
\end{enumerate}

\begin{Shaded}
\begin{Highlighting}[]
\CommentTok{\# 12a}
\NormalTok{EPA\_data\_summary }\OtherTok{\textless{}{-}}\NormalTok{ EPA\_data\_3 }\SpecialCharTok{\%\textgreater{}\%}
    \FunctionTok{group\_by}\NormalTok{(Site.Name, Month, Year) }\SpecialCharTok{\%\textgreater{}\%}
    \FunctionTok{summarise}\NormalTok{(}\AttributeTok{meanaqi\_pm =} \FunctionTok{mean}\NormalTok{(PM2}\FloatTok{.5}\NormalTok{), }\AttributeTok{meanaqi\_o3 =} \FunctionTok{mean}\NormalTok{(Ozone), }\AttributeTok{.groups =} \StringTok{"keep"}\NormalTok{)}
\FunctionTok{print}\NormalTok{(EPA\_data\_summary)}
\end{Highlighting}
\end{Shaded}

\begin{verbatim}
## # A tibble: 308 x 5
## # Groups:   Site.Name, Month, Year [308]
##    Site.Name   Month  Year meanaqi_pm meanaqi_o3
##    <chr>       <dbl> <dbl>      <dbl>      <dbl>
##  1 Bryson City     1  2018       38.9       NA  
##  2 Bryson City     1  2019       29.8       NA  
##  3 Bryson City     2  2018       27.2       NA  
##  4 Bryson City     2  2019       33.0       NA  
##  5 Bryson City     3  2018       34.7       41.6
##  6 Bryson City     3  2019       NA         42.5
##  7 Bryson City     4  2018       28.2       44.5
##  8 Bryson City     4  2019       26.7       45.4
##  9 Bryson City     5  2018       NA         NA  
## 10 Bryson City     5  2019       NA         39.6
## # ... with 298 more rows
\end{verbatim}

\begin{Shaded}
\begin{Highlighting}[]
\CommentTok{\# 12b}
\NormalTok{EPA\_data\_summary\_2 }\OtherTok{\textless{}{-}} \FunctionTok{drop\_na}\NormalTok{(EPA\_data\_summary)}
\FunctionTok{print}\NormalTok{(EPA\_data\_summary\_2)}
\end{Highlighting}
\end{Shaded}

\begin{verbatim}
## # A tibble: 101 x 5
## # Groups:   Site.Name, Month, Year [101]
##    Site.Name    Month  Year meanaqi_pm meanaqi_o3
##    <chr>        <dbl> <dbl>      <dbl>      <dbl>
##  1 Bryson City      3  2018       34.7       41.6
##  2 Bryson City      4  2018       28.2       44.5
##  3 Bryson City      4  2019       26.7       45.4
##  4 Bryson City      7  2019       33.6       30.4
##  5 Bryson City      9  2018       25.1       25.4
##  6 Bryson City     10  2018       31.3       31  
##  7 Castle Hayne     4  2018       14.9       48.7
##  8 Castle Hayne     4  2019       14.3       45.1
##  9 Castle Hayne     5  2019       16.5       42.8
## 10 Castle Hayne     7  2018       15.5       36.5
## # ... with 91 more rows
\end{verbatim}

\begin{Shaded}
\begin{Highlighting}[]
\CommentTok{\# 13}
\FunctionTok{dim}\NormalTok{(EPA\_data\_summary\_2)}
\end{Highlighting}
\end{Shaded}

\begin{verbatim}
## [1] 101   5
\end{verbatim}

\begin{enumerate}
\def\labelenumi{\arabic{enumi}.}
\setcounter{enumi}{13}
\tightlist
\item
  Why did we use the function \texttt{drop\_na} rather than
  \texttt{na.omit}?
\end{enumerate}

\begin{quote}
Answer: drop\_na() drops rows where any column specified by \ldots{}
contains a missing value. na.omit returns the object with incomplete
cases removed.
\end{quote}

\end{document}
